\documentclass{exam}
\usepackage{amsmath}
\usepackage{enumitem}
\usepackage{multicol}
\usepackage[T1]{fontenc}
\usepackage{subfig}

\begin{document}

	\begin{center}
		\fbox{\fbox{\parbox{5.5in}{\centering
				Kontrolinis darbas, 6 skyrius I grupė
				}}}
	\end{center}
	
	\vspace{5mm}
	\makebox[0.75\textwidth]{Vardas, pavardė, klasė:\enspace\hrulefill}

	\begin{questions}
	\raggedcolumns\begin{multicols}             {2} 
\question Keliais nukleinais sumažėja branduolyje vykstant alfa skilimui? \begin{choices}
    \choice 1
    \choice 0
    \choice 4
    \choice 3
\end{choices}
\question Kokia spinduliuotė labiausiai jonizuoja? \begin{choices}
    \choice Alfa.
    \choice Beta.
    \choice Gama.
    \choice Neutronai.
\end{choices}

\question Branduolinėse elektrinėse energija gaunama iš: \begin{choices}
    \choice Savaiminio branduolių skylimo.
    \choice Savaiminio branduolių jungimosi.
    \choice Grandininės branduolinės reakcijos.
    \choice Priverstinės termobranduolinės reakcijos.
\end{choices}
\question Nuo ko priklauso fotono energija? \begin{choices}
    \choice Dažnio
    \choice Greičio
    \choice Masės
    \choice Krūvio
\end{choices}
\end{multicols} 

	\question \fillin[grandininė] branduolinė reakcija yra naudojama išgauti elektros energija atominėse elektrinėse.
\question \fillin[L] yra žymimas bangos ilgis.
\question Fotonas gali įgreitinti išlaisvintus \fillin[elektronus].
\question Vykstant \fillin[beta] skylimu, branduolio protonų skaičius padidėja 1.

	\begin{figure}
        \centering
        \begin{tabular}{ |c|c|c| } 
        \hline 
metalai & energija \\ 
 \hline Li & 2.9 \\ 
 \hline Na & 2.4 \\ 
 \hline K & 2.3 \\ 
 \hline Cs & 1.9 \\ 
 \hline Ba & 2.5 \\ 
 \hline Ca & 2.9 \\ 
 \hline Nb & 2.3 \\ 
 \hline Zr & 4.05 \\ 
 \hline Mg & 3.66 \\ 
 \hline Al & 4.2 \\ 
 \hline Cu & 4.6 \\ 
 \hline Ag & 4.64 \\ 
 \hline Zn & 3.6 \\ 
 \hline Sc & 3.5 \\ 
 \hline  
 \end{tabular} 
 \end{figure} 
\begin{figure}
        \centering
        \begin{tabular}{ |c|c|c| } 
        \hline 
fotonai & energija \\ 
 \hline raudonos & 1.91 \\ 
 \hline oranžinės & 2.06 \\ 
 \hline geltonos & 2.14 \\ 
 \hline žalios & 2.25 \\ 
 \hline žydros & 2.48 \\ 
 \hline mėlynos & 2.75 \\ 
 \hline violetinės & 3.1 \\ 
 \hline  
 \end{tabular} 
 \end{figure} 

	\question Į Cs metalo plokštelę yra šviečiama žalios spalvos šviesa. Ar įvyks fotoefektas? Jei taip, kokia bus elektronų kinetinė energija?
\question Į Nb metalo plokštelę yra šviečiama žalios spalvos šviesa. Ar įvyks fotoefektas? Jei taip, kokia bus elektronų kinetinė energija?

	\question Užbaikite rašyti duotas lygtis, jei trūksta skaičių įrašykite skaičius, jei pateiktas klaustukas, 
raskite to elemento nukleonų skaičius.\begin{parts} 
\part $^{7}_{3}$Li$+$?$\longrightarrow 2^{4}_2$ He
\part $^{235}_{92}$U$\longrightarrow?+^{135}_{55}$Cs$+4^1_0n$
\part $^{250}_{96}$Cm$\longrightarrow ?+^{98}_{38}$Sr$4^1_0$n
\end{parts}
	%Q5%
	%Q6%
	\end{questions}
	
\end{document}