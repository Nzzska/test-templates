\documentclass{exam}
\usepackage{amsmath}
\usepackage{enumitem}
\usepackage{multicol}
\usepackage[T1]{fontenc}

\begin{document}

	\begin{center}
		\fbox{\fbox{\parbox{5.5in}{\centering
				 
Čia yra bandomasis kontrolinis darbas sugeneruotas Oskaro Balkaus sunkiu 2,5 dienų darbo 

prašome vertinti, gerbti ir mylėti

				}}}
	\end{center}
	
	\vspace{5mm}
	\makebox[0.75\textwidth]{Vardas, pavardė, klasė:\enspace\hrulefill}

	\begin{questions}
	\raggedcolumns\begin{multicols}             {2} 
\question Kiek nukleno sumažėja branduolyje vykstant alfa skilimui? \begin{choices}
    \choice 1
    \choice 0
    \choice 4
    \choice 3
\end{choices}
\question Iš kur atsiranda saulės skleidžiama energija? \begin{choices}
    \choice Vykstant degimui.
    \choice Vykstant branduolių dalijimuisi.
    \choice Vykstant branduolių jungimuisi.
    \choice Vykstant grandininei branduolinei reakcijai.
\end{choices}
\question Kokia radioaktyvi spinduliuotė pasižymi didžiausia skvarba? \begin{choices}
    \choice Gama spinduliuotė.
    \choice Alfa spinduliuotė.
    \choice Beta spinduliuotė.
    \choice Radijo bangos.
\end{choices}
\question Kokia yra fotono rimties masė? \begin{choices}
    \choice Tokia kaip elektrono.
    \choice Tokia kaip protono.
    \choice Begalinė.
    \choice 0.
\end{choices}
\end{multicols} 

	\question \fillin[Fotoefektas] yra reiškinys, kai fotonas išmuša elektronus iš metalo.
\question \fillin[Hz] yra dažnio matavimo vienetas.
\question Vykstant \fillin[beta] skylimu, branduolio protonų skaičius padidėja 1.
\question \fillin[J] yra energijos matavimo vienetas.

	\raggedcolumns\begin{multicols}             {2} 
\begin{center}
        \begin{tabular}{ |c|c|c| } 
        \hline 
metalai & energija \\ 
 \hline Li & 2.9 \\ 
 \hline Na & 2.4 \\ 
 \hline K & 2.3 \\ 
 \hline Cs & 1.9 \\ 
 \hline Ba & 2.5 \\ 
 \hline Ca & 2.9 \\ 
 \hline Nb & 2.3 \\ 
 \hline Zr & 4.05 \\ 
 \hline Mg & 3.66 \\ 
 \hline Al & 4.2 \\ 
 \hline Cu & 4.6 \\ 
 \hline Ag & 4.64 \\ 
 \hline Zn & 3.6 \\ 
 \hline Sc & 3.5 \\ 
 \hline  
 \end{tabular} 
 \end{center} 
\begin{center}
        \begin{tabular}{ |c|c|c| } 
        \hline 
fotonai & energija \\ 
 \hline raudonos & 1.91 \\ 
 \hline oranžinės & 2.06 \\ 
 \hline geltonos & 2.14 \\ 
 \hline žalios & 2.25 \\ 
 \hline žydros & 2.48 \\ 
 \hline mėlynos & 2.75 \\ 
 \hline violetinės & 3.1 \\ 
 \hline  
 \end{tabular} 
 \end{center} 
\end{multicols} 

	\question Į Na metalo plokštelę yra šviečiama žydros spalvos šviesa. Ar įvyks fotoefektas? Jei taip, kokia bus elektronų kinetinė energija?

	%Q4%
	%Q5%
	%Q6%
	\end{questions}
	
\end{document}