\documentclass{exam}
\usepackage{amsmath}
\usepackage{enumitem}
\usepackage{multicol}
\usepackage[T1]{fontenc}

\begin{document}

	\begin{center}
		\fbox{\fbox{\parbox{5.5in}{\centering
				 
Čia yra bandomasis kontrolinis darbas sugeneruotas Oskaro Balkaus sunkiu 2,5 dienų darbo 

prašome vertinti, gerbti ir mylėti

				}}}
	\end{center}
	
	\vspace{5mm}
	\makebox[0.75\textwidth]{Vardas, pavardė, klasė:\enspace\hrulefill}

	\begin{questions}
	\raggedcolumns\begin{multicols}             {2} 
\question Kas yra šviesos dalelė? \begin{choices}
	\choice Protonas.
	\choice Fotonas.
	\choice Elektronas.
	\choice Gama.
\end{choices}
\question Kokia yra fotono rimties masė? \begin{choices}
    \choice Tokia kaip elektrono.
    \choice Tokia kaip protono.
    \choice Begalinė.
    \choice 0.
\end{choices}
\question Kokių dalelių turi vienoda kiekį izotopai? \begin{choices}
    \choice Fotonų.
    \choice Neutronų.
    \choice Protonų.
    \choice Nukleonų.
\end{choices}
\question Energija, kurios reikia kad išlaisvinti elektroną iš metalo: \begin{choices}
    \choice Išlaisvinimo darbas.
    \choice Išlaisvinimo galia.
    \choice Elektrostatinė energija.
    \choice Branduolinė energija.
\end{choices}
\end{multicols} 

	\question \fillin[J] yra energijos matavimo vienetas.
\question \fillin[h] yra žymima Planko konstanta.
\question fotonas gali įgreitinti išlaisvintus \fillin[elektronus].
\question \fillin[protonas] - teigiamai įelektrintas nukleonas.

	%Q2%
	%Q3%
	%Q4%
	%Q5%
	%Q6%
	\end{questions}
	
\end{document}